\chapter{Einleitung}
\label{ch:introduction}
Aus der Psychologie kennen wir Ansätze wie die Event Segmentation Theory (Zacks et al. 2007), die beschreiben wie es Teil der menschlichen Wahrnehmung ist, kontinuierliche Vorgänge in diskrete, bedeutungsvolle Ereignisse zu unterteilen\cite{bib:est}. Dies ist ein Vorgang der so tief in uns verankert ist, dass alles andere uns unnatürlich vorkommt. 

Event Segmentation mit Hilfe eines Rekurrenten Neuronalen Netzes mit LSTM

Eventsegmentation mit Hilfe 


	
		Die Motivation hinter einer entsprechenden automatisierten Event Segmentation ist vielfältig. Ein möglicher, aber auch sehr ambitionierter Anwendungsbereich ist Event Segmentation von menschlichen Aktionen. Aktionen können so unterteilt, klassifiziert und eventuell sogar vorhergesagt werden. So ist zum Beispiel die Aktion des Kaffee Trinkens aus einer Tasse unterteilbar in "Hand zu Tasse führen", "Hand umgreift Henkel", "Tasse wird zum Mund geführt" und "Kaffee wird getrunken", jeweils getrennt durch Ereignisgrenzen wie z.B. "Hand erreicht Tasse". 
		
		Ein sehr einfaches Beispiel einer solchen unterteilbaren kontinuierlichen Aktivität ist das in dieser Arbeit betrachtete Bouncing Ball Szenario. Ein Ball gleitet mit steter Geschwindigkeit in einer quadratischen Ebene. Erreicht er eine Kante, prallt er von ihr ab und gleitet in die entsprechende Richtung weiter.
		
		
		
		gliederung einleitung
		kontakt zum leser aufbauen, 
		
		bedeutung neuronale netze
		Problembeschreibung und abgrenzung
		ziel: was soll herausgefunden werden
		methode und vorgehensweise
		Gliederung der Arbeit
		
	
 