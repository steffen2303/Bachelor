%%%%%%%%%%%%%%%%%%%%%%%%%%%%%%%%%%%%%%%%%%%%%%%%%%%%%%%%%%%%%%%%%%%%%%%%%%%%%
%% Beispieldatei zur Verwendung des
%% LaTeX-Style fuer das Erstellen von Abschlussarbeiten
%%%%%%%%%%%%%%%%%%%%%%%%%%%%%%%%%%%%%%%%%%%%%%%%%%%%%%%%%%%%%%%%%%%%%%%%%%%%%
%% Author(s):
%% - Michael Hoefling, hoefling@informatik.uni-tuebingen.de
%% - Mark Schmidt, mark-thomas.schmidt@uni-tuebingen.de
%%%%%%%%%%%%%%%%%%%%%%%%%%%%%%%%%%%%%%%%%%%%%%%%%%%%%%%%%%%%%%%%%%%%%%%%%%%%%

\documentclass[12pt,headsepline,oneside,ngerman]{scrreprt}

% Hier kann man weitere hilfreiche Latex-Packages einbinden
% z.B. usepackage{url}
\usepackage[utf8]{inputenc}
\usepackage[boxed]{algorithm}
\usepackage{algorithmic_sok}
\usepackage{amssymb}
\usepackage[ngerman]{babel}
\usepackage{url}
% Lehrstuhlvorlage einbinden
% Arbeitstyp als Option angeben (Diplomarbeit ist Standard), z.B.
% Bachelorarbeit: [bachelorarbeit]
% Studienarbeit: [studienarbeit]
% Masterarbeit: [masterarbeit]
% Seminarausarbeitung: [seminararbeit]
\usepackage[bachelorarbeit]{knthesis}

% hilfreiche Makros einbinden
\newcommand{\red}[1]{\textcolor{red}{#1}}
\newcommand{\green}[1]{\textcolor{green}{#1}}
\newcommand{\blue}[1]{\textcolor{blue}{#1}}


\newcommand\alg[1]{Algorithm~\ref{alg:#1}}
\newcommand\fig[1]{Figure~\ref{fig:#1}}
\newcommand\figs[2]{Figures~\ref{fig:#1}--\ref{fig:#2}}
\newcommand\twofigs[2]{Figures~\ref{fig:#1} and \ref{fig:#2}}
\newcommand\chap[1]{Chapter~\ref{ch:#1}}
\newcommand\sect[1]{Section~\ref{sec:#1}}
\newcommand\tabl[1]{Table~\ref{tab:#1}}
\newcommand\tables[2]{Tables~\ref{#1} and \ref{#2}}
\newcommand\reqn[1]{Eqn.~(\ref{eqn:#1})}
\newcommand\twoeqns[2]{Eqns. (\ref{eqn:#1}) and~(\ref{eqn:#2})}
\newcommand\reqns[2]{Eqns. (\ref{eqn:#1})--(\ref{eqn:#2})}
\newcommand\pout{P_{\rm out}}
\newcommand\appx[1]{Appendix~\ref{app:#1}}

\newcommand\LB{L\!B}
\newcommand\IB{I\!B}
\newcommand\EB{E\!B}
\newcommand\BBB{B\!B\!B}
\newcommand\ILB{I\!L\!B}
\newcommand\ELB{E\!L\!B}
\newcommand\apl{len_{path}^{avg}}
\newcommand\mpl{len_{path}^{max}}
\newcommand\NBudgetsMax{m}
\newcommand\Prob{p_a}

\newcommand\iN[2]{#1_{#2}} % indexed name
\newcommand\func[2]{#1(#2)}
\newcommand\usageBy[3]{\func{\iN{u}{#1}}{\iN{g}{#2,#3}}}
\newcommand\usageByI[4]{\func{\iN{u^{#4}}{#1}}{\iN{g}{#2,#3}}}
\newcommand\gc[2]{\func{c}{\iN{g}{#1,#2}}}
\newcommand\ga[2]{\func{a}{\iN{g}{#1,#2}}}
\newcommand\gah[3]{\func{a}{\iN{g}{#1,#2,#3}}}

\newcommand\degMaxDev{deg^{max}_{dev}}
\newcommand\degAvg{deg_{avg}}
\newcommand\degMin{deg_{min}}
\newcommand\degMax{deg_{max}}
\newcommand\Deg{deg}
\newcommand\kAvg{k^\ast}

\newcommand{\tdef}[1]{\textbf{#1}}
\newcommand{\ddef}[1]{\emph{#1}}

\providecommand{\linearelt}[1]{\mathbf{#1}}
\providecommand{\vecti}[2]{\lowercase{#1}_{#2}}
\providecommand{\mtrxi}[3]{\lowercase{#1}_{#2,\!\!\:#3}}
\providecommand{\hmtxi}[2]{\linearelt{{#1}_{#2}}}


\providecommand{\numberset}[1]{\mathbb{#1}}
\providecommand{\IC}{\numberset{C}}
\providecommand{\IN}{\numberset{N}}
\providecommand{\IQ}{\numberset{Q}}
\providecommand{\IR}{\numberset{R}}
\providecommand{\IZ}{\numberset{Z}}
\providecommand{\linearelt}[1]{\mathbf{#1}}
\providecommand{\vect}[1]{\linearelt{#1}}
\providecommand{\mtrx}[1]{\linearelt{#1}}
\providecommand{\hmtx}[1]{\linearelt{#1}}
\providecommand{\IO}{\vect{0}} \providecommand{\II}{\vect{1}}
\providecommand{\abs}[1]{\lvert #1 \rvert}
\providecommand{\norm}[1]{\lVert #1 \rVert}
\providecommand{\set}[1]{\left\{ #1 \right\}}
\providecommand{\sequ}[1]{\left( #1 \right)}
\providecommand{\path}[1]{\sequ{#1}}
\providecommand{\trace}[1]{\sequ{#1}}
\providecommand{\prop}{\colon}


\DeclareMathOperator{\tr}{tr} \DeclareMathOperator{\ti}{ti}
\DeclareMathOperator{\lb}{lb} \DeclareMathOperator{\pfx}{pfx}
\DeclareMathOperator{\sfx}{sfx}

%\newcommand\citeN\cite

\newcommand{\figeps}[3][]{%
 \begin{figure}
  \begin{center}
      %\vspace{-0.3cm}
   \leavevmode
      \parbox[t]{#1}{%
        \resizebox{#1}{!}{\includegraphics{figures/#2}}
      }
      \vspace{-0.3cm}
   \caption{#3}
   \vspace{-0.5cm}
   \label{fig:#2}
  \end{center}
 \end{figure}
}

\newenvironment{tab}[2]{%
 \begin{table}[tbh]
   \vspace{-0.35cm}
  \begin{center}
  \caption{#2\label{tab:#1}}
}{%
  \end{center}
  \vspace{-0.35cm}
 \end{table}
}

\newcommand{\twofigeps}[8]{
% first figure
% 1. parbox size 2. figure size 3. name 4. caption
% second figure
% 5. parbox size 6. figure size 7. name 8. caption
  \begin{figure}[htb]
    \leavevmode
    \begin{center}
      \parbox[t]{#1\textwidth}{%
        \resizebox{#2\textwidth}{!}{\includegraphics{figures/#3}}
        \caption{#4}\label{fig:#3}
      }
      \hfill
      \parbox[t]{#5\textwidth}{%
        \resizebox{#6\textwidth}{!}{\includegraphics{figures/#7}}
        \caption{#8}\label{fig:#7}
     }
    \end{center}
  \end{figure}
}

\newcommand{\foursubfigeps}[9]{
% first figure
% 1. parbox size 2. figure size 3. name 4. caption
% second figure
% 5. parbox size 6. figure size 7. name 8. caption
% 9. parbox size 2. figure size 3. name 4. caption
% second figure
% 13. parbox size 6. figure size 7. name 8. caption
% 17. caption of main figure
  \begin{figure}[htb]
    \leavevmode
    \begin{center}
     \subfigure[#2]{
        \label{fig:#1}
        \parbox[t]{0.47\textwidth}{%
            \resizebox{0.44\textwidth}{!}{\includegraphics{figures/#1}}
     \vspace{-1cm}
        }
     }
     \subfigure[#4]{
        \label{fig:#3}
        \parbox[t]{0.47\textwidth}{%
            \resizebox{0.44\textwidth}{!}{\includegraphics{figures/#3}}
     \vspace{-1cm}
        }
     }
     \subfigure[#6]{
        \label{fig:#5}
        \parbox[t]{0.47\textwidth}{%
            \resizebox{0.44\textwidth}{!}{\includegraphics{figures/#5}}
        }
     }
     \subfigure[#8]{
        \label{fig:#7}
        \parbox[t]{0.47\textwidth}{%
            \resizebox{0.44\textwidth}{!}{\includegraphics{figures/#7}}
        }
     }
    \end{center}
     \vspace{-1cm}
    \caption{#9}
  \end{figure}
}

\newcommand{\threesubfigeps}[7]{
% first figure
% 1. parbox size 2. figure size 3. name 4. caption
% second figure
% 5. parbox size 6. figure size 7. name 8. caption
% 9. parbox size 2. figure size 3. name 4. caption
% second figure
% 13. parbox size 6. figure size 7. name 8. caption
% 17. caption of main figure
  \begin{figure}%[htb]
    \leavevmode
    \begin{center}
     \subfigure[#2]{
        \label{fig:#1}
        \parbox[t]{0.91\columnwidth}{%
            \resizebox{0.90\columnwidth}{!}{\includegraphics{figures/#1}}
     \vspace{-0.5cm}
        }
     }
     \subfigure[#4]{
        \label{fig:#3}
        \parbox[t]{0.91\columnwidth}{%
            \resizebox{0.90\columnwidth}{!}{\includegraphics{figures/#3}}
     \vspace{-0.5cm}
        }
     }
     \subfigure[#6]{
        \label{fig:#5}
        \parbox[t]{0.91\columnwidth}{%
            \resizebox{0.90\columnwidth}{!}{\includegraphics{figures/#5}}
        }
     }
    \end{center}
    \caption{#7}
  \end{figure}
}


\newcommand{\twosubfigeps}[5]{
% first figure
% 1. parbox size 2. figure size 3. name 4. caption
% second figure
% 5. parbox size 6. figure size 7. name 8. caption
% 9. parbox size 2. figure size 3. name 4. caption
% second figure
% 13. parbox size 6. figure size 7. name 8. caption
% 17. caption of main figure
  \begin{figure}[H]
    \leavevmode
    \vspace{-0.0cm}
    \begin{center}
     \subfigure[#2]{
        \label{fig:#1}
        \parbox[t]{0.47\textwidth}{%
            \resizebox{0.44\textwidth}{!}{\includegraphics{figures/#1}}
     \vspace{-1cm}
        }
     }
     \subfigure[#4]{
        \label{fig:#3}
        \parbox[t]{0.47\textwidth}{%
            \resizebox{0.44\textwidth}{!}{\includegraphics{figures/#3}}
     \vspace{-1cm}
        }
     }
    \end{center}
      \vspace{-0.5cm}
   \caption{#5}
  \end{figure}
}



%\newcommand{\twofigeps}[8]{
%  \figeps[#1\columnwidth]{#3}{#4}
%  \figeps[#5\columnwidth]{#7}{#8}
%}


%\setlength{\headsep}{20pt} \addtolength{\topmargin}{-20pt}
%% correct bad hyphenation here
%\hyphenation{op-tical net-works semi-conduc-tor IEEEtran}

% \makeatletter
%\newcommand{\ps@myplain}{%
%  \renewcommand{\@oddhead}{\hfil COST 279 TD(03)046}%
%  \renewcommand{\@evenhead}{}%
%  \renewcommand{\@evenfoot}{}%
%  \renewcommand{\@oddfoot}{\hfil\textrm{\thepage}\hfil}%
%}
% \makeatother

\newcommand{\figepsRotate}[4][]{%
 \begin{figure}[htb]
  \begin{center}
%
%   \leavevmode
%   \if\empty#1\else\epsfxsize=#1\fi
%   %\epsfxsize=#1
%   \epsfbox{figures/#2.eps}
%
%
\includegraphics[height=#1,angle=#4]{figures/#2.eps}
   \caption{#3}
   \label{fig:#2}
  \end{center}%\vspace*{-8mm}
 \end{figure}
}

\newboolean{makevspace}
\newcommand{\cvspace}[1]{%
    \ifthenelse
        {\boolean{makevspace}}
        {\vspace{#1}}
        {}%
    }


\begin{document}

% Pfad zu Abbildungen 
\graphicspath{{figures/}}

% Konfiguration des Latex-Style laden
% In dieser Datei werden folgende Parameter der Arbeit festgelegt:
% - Autor
% - Geburtsdatum und -ort des Autors
% - Titel auf Deutsch
% - Titel auf Englisch
% - Betreuer
% - Anmeldedatum
% - Abgabedatum
% - Tiefe des Inhaltsverzeichnisses

\autor{Steffen Schnürer}
\geboren{23.03.1994}{Nagold, Baden-W\"urttemberg}

\themaenglish{}
\themadeutsch{Event Segmentation mit Hilfe eines Rekurrenten Neuronalen Netzes mit LSTM}
\assistent{Sebastian Otte}
\anmeldung{14.02.2017}
\abgabe{14.06.2017}

% Titelseiten erzeugen
%\maketitle

% Eidesstattliche Erklaerung erzeugen
\erklaerung

%% Inhaltsverzeichnis erstellen, dazu die Numerierung aendern
\pagestyle{headings}
\pagenumbering{roman}
\setcounter{secnumdepth}{3}
\setcounter{tocdepth}{2}
\renewcommand{\marginpar}[1]{}
\tableofcontents
\cleardoublepage
\pagenumbering{arabic}

\pagestyle{headings}



% Ab hier folgt der wichtige Inhalt der Arbeit.
% Die Gliederung dient nur als Vorschlag und muss je nach
% anzufertigender Arbeit angefertigt werden.
\chapter{Einleitung}
\label{ch:introduction}
Aus der Psychologie kennen wir Ansätze wie die Event Segmentation Theory (Zacks et al. 2007), die beschreiben wie es Teil der menschlichen Wahrnehmung ist, kontinuierliche Vorgänge in diskrete, bedeutungsvolle Ereignisse zu unterteilen\cite{bib:est}. Dies ist ein Vorgang der so tief in uns verankert ist, dass alles andere uns unnatürlich vorkommt. 

Event Segmentation mit Hilfe eines Rekurrenten Neuronalen Netzes mit LSTM

Eventsegmentation mit Hilfe 


	
		Die Motivation hinter einer entsprechenden automatisierten Event Segmentation ist vielfältig. Ein möglicher, aber auch sehr ambitionierter Anwendungsbereich ist Event Segmentation von menschlichen Aktionen. Aktionen können so unterteilt, klassifiziert und eventuell sogar vorhergesagt werden. So ist zum Beispiel die Aktion des Kaffee Trinkens aus einer Tasse unterteilbar in "Hand zu Tasse führen", "Hand umgreift Henkel", "Tasse wird zum Mund geführt" und "Kaffee wird getrunken", jeweils getrennt durch Ereignisgrenzen wie z.B. "Hand erreicht Tasse". 
		
		Ein sehr einfaches Beispiel einer solchen unterteilbaren kontinuierlichen Aktivität ist das in dieser Arbeit betrachtete Bouncing Ball Szenario. Ein Ball gleitet mit steter Geschwindigkeit in einer quadratischen Ebene. Erreicht er eine Kante, prallt er von ihr ab und gleitet in die entsprechende Richtung weiter.
		
		
		
		gliederung einleitung
		kontakt zum leser aufbauen, 
		
		bedeutung neuronale netze
		Problembeschreibung und abgrenzung
		ziel: was soll herausgefunden werden
		methode und vorgehensweise
		Gliederung der Arbeit
		
	
 
\chapter{Grundlagen} 
\label{ch:grundlagen}
In diesem Kapitel möchte ich einige Grundlagen für die weiteren Kapitel dieser Arbeit legen. Zunächst erläutere ich den Aufbau eines neuronalen Netzes im allgemeinen, die Bedeutung von rekurrenten neuronalen Netzen (RNN), zeige am Bouncing Ball Szenario das Vanishing Gradient Problem auf und mache so die Notwendigkeit der LSTM-Technik deutlich.
\section{Neuronale Netze}

\begin{figure}
	\centering
	\includegraphics[width=0.7\textwidth, height=150px]{pics/neuron.jpg}	
	\caption{Schematische Darstellung eines biologischen Neuron \cite{bib:neuron}}
	
	\label{img:neuron}
\end{figure}
\begin{figure}
	\centering
	
	\includegraphics[width=0.7\textwidth, height=150px]{pics/aneuron.png}	
	\caption{Schematische Darstellung eines künstlichen Neuron \cite{bib:aneuron}}
	\label{img:aneuron}
\end{figure}

Als neuronales Netz bezeichnet man eine verwobene Struktur zwischen vielen einzelnen Zellen von meistens gleichem - aber keineswegs darauf beschr\"anktem - Aufbau, den sogenannten Neuronen. Eine solche Zelle hat immer die Eigenschaft, dass sie Signale von anderen Zellen empfängt, diese gewichtet aufakkumuliert und abhängig von einer internen Aktivierungsfunktion ein entsprechendes Signal an andere Zellen weitergibt, die damit ihrerseits diesselbe Prozedur durchlaufen. Ein neuronales Netz im Gehirn einer Ameise hat ca. 250.000 Neuronen, ein menschliches 86 Milliarden (1) und wir haben lediglich eine wage Vorstellung, wozu diese imstande sind. In der Informatik werden solche Strukturen als künstliche neuronale Netze nachgebildet und je nach Problemstellung abgewandelt. Zur Vereinfachung meinen wir ab sofort, sofern nicht explizit anders angegeben, mit neuronalen Netzen künstliche neuronale Netze.
\begin{figure}
	\centering
	\includegraphics[width=0.7\textwidth, height=150px]{pics/MLP.png}	
	\caption{Ein Multilayer Perceptron mit 2 Inputneuronen(grün), einem Hidden Layer mit 5 Neuronen(blau) und einem einzelnen Outputneuron(gelb).   \cite{bib:mlp}}
	\label{img:aneuron}
\end{figure}
Es gibt viele verschiedene Arten von neuronalen Netzen, die einfachste von Ihnen ist das Multilayer Perceptron (MLP). Die Zellen werden zu mehreren Ebenen(Layer) zusammengefügt und es gibt eine geordnete Datenflussrichtung, eine Zelle erhält ihren Input von jeder Zelle der vorigenen Ebene und gibt ihren Output entsprechend an jede Zelle der folgenden Ebene weiter, Feedforward genannt (Vgl Bild 2). 
Bild 3 Künstliches Neuron 
Gl 1: xh = phi,h(net(h)) = phi,h(Sigma i€I (wih xi))
In jeder Zelle werden die Inputsignale, also die Aktivierungen der vorigen Zellen aufsummiert, aber unterschiedlich gewichtet vgl GL 1. Eben diese Gewichte sind der Kern des maschinellen Lernens, da bei einmal geschickt gefundenen Gewichten, komplexe Aufgabenstellungen und Probleme mit (vergleichsweise) wenig Rechen- sowie Programmieraufwand gelößt werden können. Im Wesentlichen versucht ein MLP immer eine Funktion zu berechnen welches einen Inputvektor der Größe n auf einen Outputvektor der Größe m abbildet. 
FM: R^n -> R^m (2 Seite 11). 
Ein Beispiel für eine solche Funktion wäre für ein gegebenes Bild zu entscheiden, ob darauf ein Gesicht zu erkennen ist. Die Länge des Inputs wäre hier z.B. mehrere Millionen, einfach die Pixelwerte, als Output wäre hier nur eine Zahl, nämlich ob ein Gesicht zu sehen ist oder nicht. Der Output y des Netzes ist also abhängig vom Input x, aber auch von den Gewichten w.
Die Gewichte werden entweder mit Hilfe von Trainingsdaten, bestehend aus Inputdaten und entsprechenden Outputdaten, also den zugehören Lösungen, oder mit einer zu erlenenden Zielfunktion, die einem aus gegebenen Inputdaten die gewünschten Lösungen liefert, über mehrere Trainingsläufe (Epochen) justiert. Fügt man in das Input-Layer entsprechende Inputdaten x ein, berechnet die Aktivierungen aller Zellen, vergleicht die Aktivierungen y des Outputlayers des Netzes mit den gegebenen Lösungen z und erhält eine Fehlerabweichung E(z,y) (GL2). 
Gl 2: E(z,y) =def 1/2 Sigma.i=1.m (zi-yi)²
Die Herausforderung ist es nun, diejenigen Gewichte w zu finden, für die die Fehlerabweichung über alle Trainingsdaten minimal ist.
GL 3: arg min w sigma(x,z)€trainset E(z,fM(w,x)).
Geläufig ist hier der Backpropagation-Algorithmus. 
Code einfügen? 
\section{Rekurrente Neuronale Netze}

Neuronale Netzwerke haben einen für einen 1-Dimensionalen Input einen klar definierten 1-Dimensionalen Output und für die meisten Problemstellungen ist auch genau diese Kompetenz gefordert. Es gibt aber auch Probleme, bei denen der Output nicht nur von diesem, sondern auch von allen beziehungsweise einigen vorigen Inputs / Zuständen abhängt. Ein anschauliches Beispiel wäre hier die Klassifizierung von Videoausschnitten, wo die Bedeutung eines einzelnen Frames vom Kontext der vorigen Frames abhängt, hingegen ein einfacheres Beispiel wäre das in dieser Arbeit betrachtete Bouncing Ball Szenario. Befindet sich ein Ball im Punkt (0/0) und der nächste Ort soll vorhergesagt werden, ist es natürlich entscheidend ob der Ball sich zuvor beispielsweise im Punkt (0.1/0.1) oder im Punkt (-0.1/-0.1) befand. Hierzu werden den Zellen des neuronalen Netzes, außer den Input und Output Neuronen rekurrente (von lat. badidadum) Verbindungen hinzugefügt, wie z.B. In Abbildung 43.
Bild eines rekurrenten Netzes
Eine Zelle bekommt nun seinen Input immernoch von allen Zellen des unteren Layers, zusätzlich nimmt sie aber auch noch als Input den Output aller Zellen desselben Layers, aber aus dem vorigen Zeitschritt. 
Formel für aktivierungen in RNN
Da diese neuen rekurrenten Inputs auch wieder gewichtet verarbeitet werden, müssen diese erst noch geschickt gefunden werden. Dies macht man mit Backpropagation through time (Rückpropagierung durch die Zeit, BPTT). Dies passiert analog zur Backpropagation im MLP, jedoch werden die einzelnen Zeitschritte aufgeklappt und (entnehme Formulierung aus Literatur und zitiere sie hier) 
Formul für BPTT einfügen.

\section{Das Vanishing Gradient Problem}

%\chapter{Implementation}
Nun da die Grundlagen gelegt sind können wir anfangen zu untersuchen wie ein LSTM Netz mit Events umgeht. Wir betrachten dazu das Bouncing Ball Szenario, da es klare Events hat und sich dadurch auch gut segmentieren lässt. Zur Implementation wurde JANNLab (Otte et al. 2013), ein Java Framework für Neuronale Netze verwendet \cite{bib:jannlab}. Dieses Kapitel soll die im Zuge dieser Arbeit verwendeten Techniken erklären und aufzeigen wie die im nächsten Kapitel beschriebenen Ergebnisse erreicht wurden, sowie welche Probleme sich dabei in den Weg gestellt haben.

\section{Das Bouncing Ball Szenario}
Das Bouncing Ball Szenario beschreibt einen Ball, der gleichförmig durch die Ebene gleitet bis er an entsprechenden Begrenzungen abprallt. Wir haben die 1- und die 2-dimensionale Version verwendet.

Im 1D Fall haben wir einen Ball ohne Ausdehnung (also eigentlich ein Punkt aber diese Unterscheidung ist für unsere Zwecke irrelevant) mit einer Position $ x_{b} $ und einer konstanten Geschwindigkeit $  v$. Dieser springt zwischen den Grenzen $ x_{1}=-1 $ und $ x_{2}=1 $ hin und her, also immer wenn die Position des Balls eine der Grenzen annimmt, wird die Geschwindigkeit invertiert $ v := -v $. Also genau so wie man es sich vorstellt. Der 1D Fall unterteilt sich also in 2 Ereignisse, die "Bewegung nach links" und die "Bewegung nach rechts", wenn man das Bouncen zwischen -1 und 1 als Bewegung auf einer horizontalen Linie nimmt.

Der 2D Fall funktioniert analog, der Ball hat eine Position $ (x_{b}/y_{b}) $, eine konstante Geschwindigkeit $ (v_{x}/v_{y}) $, welcher nun zwischen den 4 Grenzen $ x_{1}=-1 $ und $ x_{2}=1 $, $ y_{1}=-1 $ und $ y_{2}=1 $ hin und her springt. Nimmt nun eine Koordinate des Balls eine der Grenzen an wird die entsprechende Koordinate invertiert, also $ (v_{x}/v_{y}) = (-v_{x}/v_{y}) $ beziehungsweise $ (v_{x}/-v_{y}) $. Der 2D Fall unterteilt sich also in 4 Ereignisse, die "Bewegung in Richtung oben-rechts", etc. wenn man das Bouncen zwischen dem Quadrat von (-1/-1) und (1/1) im 2-dimensionalen als Bewegung in einer stehenden Ebene nimmt.

Das 1D-Szenario habe ich hauptsächlich verwendet um mich einzuarbeiten und zurechtzufinden, da es unter anderem mit weniger Neuronen erlernt werden kann und so die Aktivierungen der einzelnen Gates auch übersichtlicher sind. Der Grund warum das 1D-Szenario auch für diese Ausarbeitung wichtig ist, ist das unter Hinzunahme der Zeitachse aussagekräftige Diagramme erstellt werden können, anhand derer einige Trainingseffekte und Verläufe veranschaulicht werden können. Entsprechende Analysen konnten vom 2D-Fall auch gemacht werden, wenn man sah wie sich eine Linie oder ein Ball mit Verzögerung entsprechend über den Bildschirm bewegt, die resultierenden Bilder sind jedoch eher unübersichtlich. Das 2D Szenario hingegen ist das für unsere Untersuchungen interessantere, da es mehr Events gibt, die auch in komplexernen Beziehungen zueinander stehen. In Abbildung \ref{img:1dvs2d} sieht man jeweils 2 Beispiele, zu dem genauen Aufbau des dazu benutzten LSTM Netzes kommen wir in der folgenden Sektion.
\begin{figure}
	\centering
	\includegraphics[width=0.9\textwidth, height=370px]{pics/1dvs2d.jpg}	
	\caption{4 Versionen des Bouncing Ball Szenario erlernt durch LSTM Netze. Zu sehen sind der Netzinput (grün), der Netzoutput (rot) und das Trainingsziel (blau). Links sieht man 2 Beispiele des 1D Falls, zur besseren Veranschaulichung sind die Daten nach den Zeitschritten vertikal versetzt. Rechts sieht man 2 Beispiele des 2D Falls, die Abbildung zeigt jeweils den Verlauf des Balls bis kurz vor dem vollenden einer Runde, um Überlagerungen in der Abbildung zu vermeiden. Der Startpunkt ist durch den schwarzen Punkt gekennzeichnet. In den jeweils oberen Fällen wurde als Netzinput die Position des Balles und als Trainingsziel die Geschwindigkeit des Balles für den nächsten Schritt. In den jeweils unteren Fällen wurde als Netzinput die Position des Balles und als Trainingsziel die vorauszuberechnende nächste Position des Balles.}
	\label{img:1dvs2d}
\end{figure}

\section{Unser LSTM, Parameter und Testfälle}
Für unser LSTM-Netz haben sich folgende Parameter als sinnvoll erwiesen:
\begin{description}	\item[Aktivierungsfunktionen:]\hfill \\ 
	Zur Ansteuerung der Gates der LSTM-Zellen im Hiddenlayer werden Neuronen mit \textit{Tangens hyperbolicus} als Aktivierungsfunktion verwendet. Für das Outputlayer wird zur Aktivierung die lineare Funktion verwendet, da die Position des Balles gleichverteilt Werte zwischen -1 und 1 annimmt, soll unser Netzoutput dasselbe tun. Anfangs war für das Outputlayer per Defaultoption die \textit{sigmoid}-Funktion zur Aktivierung eingestellt, deren Bild aber nur von 0 bis 1 reicht. Das hatte natürlich nicht funktioniert und für Verwirrung gesorgt, warum denn nur die Hälfte des Problems erlernt wird.  
	\item[Trainingsoptimierung:]\hfill \\ 
	Als Optimierungsmethode für das Traing wurde die Adaptive Moment Estimation Methode(ADAM), also eine adaptive Momentumsschätzung, verwendet. Diese baut die übliche Lernmethode mit Lernrate und Momentumrate dahingehend aus, dass diese durchgehend angepasst werden, um noch schneller zu genaueren Ergebnissen zu kommen. Hierzu wurden die Standard Parameter verwendet, also $ \beta_{1}=0.9, \beta_{2}=0.99 $ und $ \epsilon = 10^{-8}$. Wobei $ \beta_{1} $ und $ \beta_{2} $ die Abklingraten des Einflusses der vergangenen Momenti auf den jetztigen Zeitschritt sind. Sind sie Nahe 1, klingt der Einfluss langsamer ab. $ \epsilon$ ist ein glättender Term, der Division durch 0 verhindert. \cite{bib:adam} Es wurde für Debugging-Zwecke an diesen Werten herumgespielt, es wurden aber nie sichtlich bessere Ergebnisse erzielt.
	\item[Netzgröße:]\hfill \\ 
	Die passende Netzgröße ist  fallabhängig, in Abbildung \ref{img:1dvs2d} wurden für die 1D Fälle 1-2-1 Netze verwindet (also 1 Inputneuron, 1 Hiddenlayer mit 2 Neuronen und 1 Outputneuron), für die 2D Fälle 2-4-2 Netze. An anderer Stelle werden andere Netzgrößen verwendet, dann wird das an entsprechender Stelle auch nochmal aufgegriffen.
	\item[Außerdem:]\hfill \\
	Alle Layer sind mit Bias-Unit, um mit möglichst wenig Neuronen möglichst viel Funktionalität zu erreichen, damit deren Analyse wiederrum möglichst ausgiebig verläuft. Peepholes \cite{bib:lstm2} wurden getestet, hatten in diesem Beispiel keinen sichtbaren Effekt auf das Training und wurden dann weggelassen.  
\end{description}
Außerdem werden verschiedene Testfälle mit verschiedenen Intentionen betrachtet, die ich an dieser Stelle vorgreifend definieren möchte. Es wurde mit fast allen Kombinationen experimentiert:
\begin{description}
	\item[Verschiedene Inputs:]
	\begin{itemize}
		\item Position: Der Standard Input
		\item Noicy Position: 
		\item Null: 
		\item Geschwindigkeit:
	\end{itemize}
	Der Standart Input in das Netz ist die Position des Balles. 
	\item[Verschiedene Trainingsziele:]
	\item Nächste Position: Der Standard Output.
	\item Geschwindigkeit: Da die Geschwindigkeit bis auf Vorzeichen konstant ist,  kann sich das Netz hier darauf konzentrieren, in welchem Ereignis, also Bewegungsabschnitt es sich befindet, also ob sich der Ball momentan z.B. nach links oben bewegt.  
	\item Nächstes Event: Es wurde auch getestet wie sich das Netz verhält wenn es selbst vorhersagen soll, welches Event als nächstes ansteht. War erfolgreich auch bei zufälligen Startposition und Geschwindigkeiten, was aber nicht weiter Überraschend ist, das sich das LSTM Netz auf ein lineares Problem, welche Grenze als nächstes getroffen wird, trainieren lässt. Auch aus den Aktivierungen im Netz konnte nichts interessantes gefolgert werden, weswegen das nächste Event als Trainingsziel nicht mehr aufgegriffen wird. 
	\item[Startposition und Geschwindigkeit:]\hfill \\
	
	\item[Feedback:]\hfill \\
\end{description}


\section{Das Training des Bouncing Verhaltens}

\section{Umfragen zum Thema}
1 von 1 befragten Personen mit einem Abschluss in Bachelor of Science meinten zum Thema: \\
Davon weiß ich nichts. \cite{bib:lstm}


\section{Feedback}



Das Training des Bouncing Verhaltens
Das interessante am Bouncing Ball Szenario ist natürlich nicht die gleichförmige Bewegung, sondern das nicht-lineare Verhalten beim Abprallen. Dieses wollen wir unserem LSTM Netzwerk möglichst genau erlernen und haben dazu einige Techniken angewendet. 

Reaktives vs vorhersehendes Verhalten
1D mit 1b Zelle

RNN Annäherung vs echte nichtlinearität
Random vs nicht Random, nur periodisch gelernt
Feedback für random, erklären warum es nicht funkltioniert hat,
Feedback, Zittern erklären für Input außerhalb des eigenen Raums
Schritte damit es funktioniert, feedback annäherung, hat nicht geholfen
Random diskretisiert um numerisch genau definiertes Rand verhalten zu haben
%\chapter{Gates und Eventgrenzen}
\label{ch:untersuchung}
Nun haben wir also Neuronale Netz mit LSTM Speicherzellen auf verschiedene Fälle des Bouncing Ball Szenarios trainiert. Wenn man dabei die internen Aktivierungen der Gates über die Zeit betrachtet und vergleicht mit dem Verlauf der Ereignisse der trainierten Aktion, findet man Zusammenhänge. In diesem Kapitel werden wie erst Beispiele für solche Zusammenhänge betrachten, dann schauen wie man diese automatisiert finden kann. Zum Schluss schauen wir noch, wie wir mit Hilfe von Backpropagation through time Vorhersagen über Events anhand der Gate-Aktivierungen treffen können und wir dieselbe Technik nutzen können um die Eventbeziehungen zu klassifizieren.
\section{Gate-Aktivierungen}
Die Möglichkeiten wie sich eine LSTM über die Zeit verhalten kann sind zahlreich, noch zahlreicher die Möglichkeiten wie mehrere von ihnen miteinander interagieren. \cite{bib:graves}. 





\begin{figure}
	\centering
	\includegraphics[width=\textwidth, height=400px]{pics/act1.png}	
	\caption{TODO }
	\label{img:act1}
\end{figure}
\begin{figure}
	\centering
	\includegraphics[width=\textwidth, height=400px]{pics/act2.png}	
	\caption{TODO }
	\label{img:act2}
\end{figure}
\begin{figure}
	\centering
	\includegraphics[width=\textwidth, height=400px]{pics/act3.png}	
	\caption{TODO }
	\label{img:act3}
\end{figure}
\begin{figure}
	\centering
	\includegraphics[width=\textwidth, height=400px]{pics/act4.png}	
	\caption{TODO }
	\label{img:act4}
\end{figure}







\section{Automatisierte Klassifizierung}
\begin{table}
\centering
\begin{tabular}{|c|c|c||c|c|c|}

	\hline
	\multicolumn{6}{|c|}{O = A+0.5B; P = B OR (1-C); Q = A XOR C} \\	\hline
	A & B & C & O & P & Q \\ \hline
	0 & 0 & 0 & 0 & 1 & 0\\ \hline
	0 & 0 & 1 & 0 & 0 & 1\\ \hline
	0 & 1 & 0 & 0.5 & 1 & 0\\ \hline
	0 & 1 & 1 & 0.5 & 1 & 1\\ \hline
	1 & 0 & 0 & 1   & 1 & 1\\ \hline
	1 & 0 & 1 & 1   & 0 & 0\\ \hline
	1 & 1 & 0 & 1.5 & 1 & 1\\ \hline
	1 & 1 & 1 & 1.5 & 1 & 0\\ \hline \hline
	\multicolumn{6}{|c|}{Backpropagation für O:=-1} \\ \hline
	$\sum \delta_{A} $ & $\sum \delta_{B} $ &$\sum \delta_{C} $ & Q & P & Q \\ \hline
	2.64 & 1.35 & -1.5*$ 10^{-4} $ & -1 & P & Q \\ \hline
	\multicolumn{6}{|c|}{Backpropagation für P:=-1} \\ \hline
	$\sum \delta_{A} $ & $\sum \delta_{B} $ &$\sum \delta_{C} $ & O & P & Q \\ \hline
	9*$ 10^{-4} $& 0.07 & -0.07 & O & -1 & Q \\ \hline
	\multicolumn{6}{|c|}{Backpropagation für Q:=-1} \\ \hline
	$\sum \delta_{A} $ & $\sum \delta_{B} $ &$\sum \delta_{C} $ & O & P & Q \\ \hline
	0.81 & $ 2*10^{-3} $ & 0.81 & O & P & -1 \\ \hline

\end{tabular}
	\caption{Krasse Tabelle}
\end{table}
\begin{figure}
	\centering
	\includegraphics[width=0.6\textwidth, height=160px]{pics/diagramm.png}	
	\caption{TODO }
	\label{img:diagramm}
\end{figure}
\begin{table}
	\centering
\begin{tabular}{|c|c|c|c|c|}
	\hline
	\multicolumn{5}{|c|}{Input Gate} \\	\hline
	Zellennummer & LO & RO & RU & LU \\ \hline
	1 & 0.049 & 13 & -16 & 0.11 \\ \hline
	2 & -0.77 & -15 & 20 & -0.12 \\ \hline
	3 & 2.1 & 7.7 & -15 & 0.04 \\ \hline
	4 & 0.11 & -15 & 18 & 0.11 \\ \hline
	5 & -1.2 & -27 & 36   & -0.080 \\ \hline
	6 & -0.89 & -5.8 & 9   & -0.02 \\ \hline\hline
	\multicolumn{5}{|c|}{Forget Gate} \\	\hline
	Zellennummer & LO & RO & RU & LU  \\ \hline
	1 & 0.046 & 0.30 & -0.16 & 0.017 \\ \hline
	2 & 0.048 & 0.014 & 0 & -0.045 \\ \hline
	3 & 0.071 & -0.14 & -0.33 & 0.20 \\ \hline
	4 & -0.20 & -1.14 & 0.92 & 0.16 \\ \hline
	5 & 0.11 & -0.05 & -0.21   & -0.12 \\ \hline
	6 & -0.18 & 0.58 & 0.070   & 0.015 \\ \hline\hline
	\multicolumn{5}{|c|}{Output Gate} \\	\hline
	Zellennummer & LO & RO & RU & LU \\ \hline
	1 & -5.9 & 2.0 & 7.2 & -0.20 \\ \hline
	2 & 10 & -14 & 20 & 0.54 \\ \hline
	3 & 8.5 & -5.1 & -3 & 0.33 \\ \hline
	4 & -8.2 & 11 & -18 & -0.40 \\ \hline
	5 & -7.3 & 8.4 & -9,6   & -0.37 \\ \hline
	6 & -2.5 & -0.55 & 7.9   & -0.077 \\ \hline\hline
\end{tabular}
\caption{Krasse Tabelle}
\end{table}
\begin{figure}
	\centering
	\includegraphics[width=\textwidth, height=200px]{pics/inputgate.png}	
	\caption{TODO }
	\label{img:inputgate}
\end{figure}
\begin{figure}
	\centering
	\includegraphics[width=\textwidth, height=200px]{pics/forgetgate.png}	
	\caption{TODO }
	\label{img:forgetgate}
\end{figure}
\begin{figure}
	\centering
	\includegraphics[width=\textwidth, height=200px]{pics/outputgate.png}	
	\caption{TODO }
	\label{img:outputgate}
\end{figure}
\section{Analyse mit BPTT}

periodisch: sieht man immer, das nicht das problem, sogar klar. overfitting

Verhalten an der Eventgrenze:

runde annäherung nicht erwünscht, aber wird es immer sein nur eben immer unrunder. im besten fall, dauert die eventgrenze immernoch über 4 zeitschritte an(1D-Fall), was durch Diskretisierung abgehackt aussieht, aber doch eine kurve ist bei kontinuierlich




RNN Annäherung vs echte nichtlinearität

anders als angenommen: zufall war garnicht wichtig. Alles was bestätigt wurde, 2 1D Fälle überlagert

dieser fall den man bekommen hätte mit feedback und zufällig und perfekt trainiert, vllt sogar viel langweiliger, da event tatsächlich nur von position des Balles abhängt, nur ein einzelnes Gewicht, wenn 1 erreicht ausgelöst ohne interaktion. 


mit backpropagation im 1D fall, sieht man: Lstm zelle forgetgate ist auf 1. wir bewegen uns nach links. wenn man gradient +1 wählt, kommt dann auf selbe zelle negativer gradient zurück? also um wieder nach links zu gehen muss man erstmal nach rechts gehen. 

Wunschergebniss, falls man es nicht sieht beschriebe es un erzähle warum man es nicht sieht: 
Hat ein Netz ein Event gut gelernt, merkt man das auch anhand der Zellen die eventeinteilung klappt besser.
Dierekter zusammenhang: lstm netz gut trainiert, mlp zur est gibt gute ergebnisse.

Wunschergebniss
Unterschiede zwischen random und nicht random
Bei random funktioniert est viel besser,  da  durch unterschiedliche geschwindigkeiten die eventkanten besser gelernt werden müssen, bzw bei nicht random werden periodische funktionen gelernt die die event segmentation behindern.

Vergleiche: Testfälle Random 1 und Random 2 Richtungen. Inwiefern weisen die LSTM zellen hinweise auf eventsegmentation? 

Vergleiche: 2D mit 6-8 und 16 zellen gelernt.
Und mit 4? Was ist die kritische Zahl?

Erstelle eine Ansicht, die anzeigt welche Zellen und Werte genutzt wurden.

Stelle 2 Hypothesen auf was den in den Gates untersucht wird, ist es ob ich mich grad nach oben rechts bewege oder das oben und links die letzten Events stattgefunden haben.

Teste 2 verschiedene mlp formen: 4 outputs für wahrscheinlich des nächsten events,
oder 2 outputs für wahrscheinlich das jeweils konträre die nächsten sind.


Vorzeigefall: Rekurenz grundding: Gleicher Input kann abhängig von vorigen stats unterschiedlichen outputs leifern, man befindet sich dank lstm auch in unterschiedlichen gatepositionen.
Zeige ball der in 0,0 sitz der nach unten rechts un nach oben links fliegt und die aktivierungen

Feedback sollte integriert werden, um die eventtransition mehr zu werden. Darlegen warum event sauberer is wenn feedback benutzt ist. Vllt hätte das auch zu klareren events geführt. Für klarrerer events wurde auch neues random entwickeld bei dem die kante bei jedem step sauber getroffen wird.

Kann MLP abschätzen wie Nahe das Event ist?
Verschiedene Thresholds, Event in 2 Steps, 5 Steps, 10 Steps, Alarm geben wenn Kollision bevorsteht.

Erklären wie die Trainingsdaten für das mlp aussehen

Vergleiche mlp das next und last event sagt
Vorhersage: last event exakt deutlich, next event unklarer das ob oben oder rechts als nächstes getroffen wird vom vall abhängt wenn man von links unten kommt-

testcode:
est: prediction error löst event unterteilung aus. würde man sich den prediction error mit ausgeben lassen. ausgabe soll diskrepanz zwischen target und output der anderen vorhersagen. Oder aber output-target diskrepanz dem input hinzufügen, damit das netz darauf arbeiten kann. 

Wichtig:
Vergleiche im LSTM die bedeutung der einzelnen Gates auf EST. Zentral

%\chapter{Ergebnisse}
\label{ch:results}


%\chapter{Zusammenfassung und Ausblick} 
\label{ch:ende}
leer...
\section{Zusammenfassung}
VAus EST Wir haben im LSTM diese Bottum up, Fehler die zur Eventunterteilung beitragen, cool wäre eine Top Down Unterteilung, wir wollen also dort hin, wie schaffen wir das. Wir sind da und wollen dahin. Das haben wir in dieser Arbeit leider nicht erreicht.

Anwendung? Hat man ein problem mit lstm erlernt und möchte nachträglich sehen ob man an den gates Events erkennt? Oder möchte man das lstm wirklich so erlernen damit events daran erkannt werden können? 

\section{Ausblick}
leer...
%\chapter{Anhang} 
\label{ch:appendix}

\newpage
%\chapter{Glossar} 
\label{ch:glossar}
	\begin{itemize}
		\item \textbf{Neuronales Netz (NN):}\\
		geiles Zeug
		\item \textbf{Rekurrentes neuronales Netz (RNN):}\\
		geileres Zeug
		\item \textbf{Gradient:}\\
		geileres Zeug
		\item \textbf{In- Output:}\\
		Zeug
		\item \textbf{LSTM:}\\
		Zeug
		\item \textbf{Forwardpass:}\\
		Zeug
		\item \textbf{Backpropagation:}\\
		Zeug
		\item \textbf{Backpropagation through time BPTT:}\\
		Zeug
		\item \textbf{Vanishing gradient problem (VGP):}\\
		Zeug
		\item \textbf{Hiddenlayer:}\\
		Zeug
		\item \textbf{Forget-,Output- und Inputgate:}\\
		Zeug
		\item \textbf{Bottum up/Top Down:}\\
		Zeug
		\item \textbf{Event/Ereignis:}\\
		Zeug
		\item \textbf{ADAM:}\\
		Zeug
		\item \textbf{Gate:}\\
		Zeug
		\item \textbf{Eventgrenze:}\\
		Zeug
		\item \textbf{:}\\
		Zeug
		
		
		
	
	\end{itemize}


% Ab hier kommt der Anhang (optional)
%\appendix
%\input{chapters/11-appendix}

% Literaturverzeichnis
\cleardoublepage
\begin{thebibliography}{xxxx}
\bibitem{bib:neuron}\url{online.science.psu.edu/bisc004_activewd001/node/1907} (abgerufen am 10.05.2017)
\bibitem{bib:aneuron}\url{de.wikipedia.org/wiki/K%C3%BCnstliches_Neuron} (abgerufen am 10.05.2017)
\bibitem{bib:mlp}\url{de.wikipedia.org/wiki/Neuronales_Netz#/media/File:Neural_network.svg} (abgerufen am 10.05.2017)
\bibitem{bib:number}\url{en.wikipedia.org/wiki/List_of_animals_by_number_of_neurons} (abgerufen am 20.05.2017)
\bibitem{bib:nn}\url{www.neuronalesnetz.de/} (abgerufen am 02.06.2017)
\bibitem{bib:bp}\url{www.informatikseite.de/neuro/node24.php} (abgerufen am 03.06.2017)
\bibitem{bib:rnn}\url{www.wildml.com/2015/09/recurrent-neural-networks-tutorial-part-1-introduction-to-rnns/} (abgerufen am 04.06.2017)
\bibitem{bib:vgp}\url{http://eric-yuan.me/rnn2-lstm/} (abgerufen am 14.05.2017)
\bibitem{bib:c}\url{aa} (abgerufen am 10.05.2017)


\end{thebibliography}
\renewcommand{\bibname}{Literaturverzeichnis}
\bibliographystyle{abbrv}
%\bibliography{literature}

% Abbildungsverzeichnis (optional)
\cleardoublepage
\addcontentsline{toc}{chapter}{Abbildungsverzeichnis}
\listoffigures
% erst gleichung nennen und dann erklären oder erst beschreiben und dann gleichung
\end{document}

